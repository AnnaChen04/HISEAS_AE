\usepackage{setspace}
\usepackage[left=3cm,right=6cm,top=2cm,bottom=3cm]{geometry}
\onehalfspacing

%===================================basic_math_packages======================================%
\usepackage{amsmath}
\usepackage{amssymb}
\usepackage{amsthm}

%Theorem environments
\newtheorem{definition}{Definition}[section]
\newtheorem{lemma}{Lemma}[section]
\newtheorem{thm}{Theorem}[section]
\newtheorem{ex}{Example}[section]
\newtheorem{persp}{Perspective}[section]
\newtheorem{tip}{Tip}[section]
%===================================margin_notes======================================%
\usepackage{ragged2e} %for \justifying
\usepackage{marginnote}
\renewcommand*{\raggedleftmarginnote}{}
\renewcommand*{\marginfont}{\scriptsize\color{gray}\sffamily\justifying}

%===================================drawing_tikz======================================%
\usepackage{tikz}
\usetikzlibrary{arrows.meta, positioning, shapes, fit, calc,matrix}
\usepackage{pgfplots}
\pgfplotsset{compat=1.18}

%===================================algorithm boxes and algorthimicx======================================%
\usepackage[most]{tcolorbox} %most for a more advanced functionalities like enhanced and breakable
\usepackage{algorithmicx}
\usepackage{algpseudocode}


%defining tcolorbox algobox
\newtcolorbox{algobox}[2][]{%
  enhanced,
  breakable,
  colback=white,
  colframe=black,
  fonttitle=\bfseries,
  title={Algorithm: #2},
  sharp corners,
  width=0.85\textwidth,  % Adjust width here
  boxrule=0.8pt,
  #1
}

\usepackage{xcolor}
\definecolor{darkpuple}{HTML}{6a537d}
\definecolor{lightpuple}{HTML}{d6ccdb}
\definecolor{rose}{HTML}{c4959b}
\definecolor{beige}{HTML}{d9cece}
\tcbset{
    ex/.style= {enhanced,
        colback=lightpuple,
        leftrule=3mm,
        rightrule=0pt,
        toprule=0pt,
        bottomrule=0pt,
        sharp corners,
        boxrule=0pt,
        width=\linewidth,
        underlay={
            \path[fill=darkpuple]
            ([xshift=0pt,yshift=1.1mm]frame.north west) --
            ([xshift=37mm,yshift=1.1mm]frame.north west) --
            ([xshift=36mm,yshift=-1mm]frame.north west) --
            ([xshift=0pt,yshift=-1mm]frame.north west) -- cycle;
        }
    },
    thm/.style= {enhanced,
        colback=beige,
        leftrule=3mm,
        rightrule=0pt,
        toprule=0pt,
        bottomrule=0pt,
        sharp corners,
        boxrule=0pt,
        width=\linewidth,
        underlay={
            \path[fill=rose]
            ([xshift=0pt,yshift=1.1mm]frame.north west) --
            ([xshift=37mm,yshift=1.1mm]frame.north west) --
            ([xshift=36mm,yshift=-1mm]frame.north west) --
            ([xshift=0pt,yshift=-1mm]frame.north west) -- cycle;
        }
    }
}


%===================================fancy header======================================%
\usepackage{fancyhdr}
\usepackage{titling} %for author
\author{Haoyu Tang}
\fancypagestyle{plain}{%  the preset of fancyhdr 
    \fancyhf{} % clear all header and footer fields
    \renewcommand{\headwidth}{15.3cm}
    \fancyhead[R]{\theauthor}
    \fancyhead[L]{Mathematithod}
}
%=====================================================
\usepackage{xcolor}
\definecolor{colbackblue}{RGB}{212,215,250}

\usepackage{float}
\usepackage{wrapfig}